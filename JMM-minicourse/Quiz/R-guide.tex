%% Author: Daniel Kaplan
%% Subject: R basics
%% Title: Study Guide to R Basics

% !Rnw weave = knitr

\documentclass[10pt]{article}\usepackage[]{graphicx}\usepackage[]{color}
%% maxwidth is the original width if it is less than linewidth
%% otherwise use linewidth (to make sure the graphics do not exceed the margin)
\makeatletter
\def\maxwidth{ %
  \ifdim\Gin@nat@width>\linewidth
    \linewidth
  \else
    \Gin@nat@width
  \fi
}
\makeatother

\definecolor{fgcolor}{rgb}{0.345, 0.345, 0.345}
\newcommand{\hlnum}[1]{\textcolor[rgb]{0.686,0.059,0.569}{#1}}%
\newcommand{\hlstr}[1]{\textcolor[rgb]{0.192,0.494,0.8}{#1}}%
\newcommand{\hlcom}[1]{\textcolor[rgb]{0.678,0.584,0.686}{\textit{#1}}}%
\newcommand{\hlopt}[1]{\textcolor[rgb]{0,0,0}{#1}}%
\newcommand{\hlstd}[1]{\textcolor[rgb]{0.345,0.345,0.345}{#1}}%
\newcommand{\hlkwa}[1]{\textcolor[rgb]{0.161,0.373,0.58}{\textbf{#1}}}%
\newcommand{\hlkwb}[1]{\textcolor[rgb]{0.69,0.353,0.396}{#1}}%
\newcommand{\hlkwc}[1]{\textcolor[rgb]{0.333,0.667,0.333}{#1}}%
\newcommand{\hlkwd}[1]{\textcolor[rgb]{0.737,0.353,0.396}{\textbf{#1}}}%

\usepackage{framed}
\makeatletter
\newenvironment{kframe}{%
 \def\at@end@of@kframe{}%
 \ifinner\ifhmode%
  \def\at@end@of@kframe{\end{minipage}}%
  \begin{minipage}{\columnwidth}%
 \fi\fi%
 \def\FrameCommand##1{\hskip\@totalleftmargin \hskip-\fboxsep
 \colorbox{shadecolor}{##1}\hskip-\fboxsep
     % There is no \\@totalrightmargin, so:
     \hskip-\linewidth \hskip-\@totalleftmargin \hskip\columnwidth}%
 \MakeFramed {\advance\hsize-\width
   \@totalleftmargin\z@ \linewidth\hsize
   \@setminipage}}%
 {\par\unskip\endMakeFramed%
 \at@end@of@kframe}
\makeatother

\definecolor{shadecolor}{rgb}{.97, .97, .97}
\definecolor{messagecolor}{rgb}{0, 0, 0}
\definecolor{warningcolor}{rgb}{1, 0, 1}
\definecolor{errorcolor}{rgb}{1, 0, 0}
\newenvironment{knitrout}{}{} % an empty environment to be redefined in TeX

\usepackage{alltt}
\usepackage{fancyhdr}
\usepackage{graphicx, verbatim}
\setlength{\textwidth}{6.5in} 
\setlength{\textheight}{9in}
\setlength{\oddsidemargin}{0in} 
\setlength{\evensidemargin}{0in}
\setlength{\topmargin}{0cm}

\pagestyle{fancy}

\lhead{\textsc{Prof. Horton}}
\chead{\textsc{STAT 135: \texttt{R} Study Guide}}
\rhead{\textsc{Fall 2014}}
\lfoot{}
\cfoot{}
\cfoot{\thepage}
\rfoot{}
\renewcommand{\headrulewidth}{0.2pt}
\renewcommand{\footrulewidth}{0.0pt}
\IfFileExists{upquote.sty}{\usepackage{upquote}}{}
\begin{document}
%\SweaveOpts{concordance=TRUE}


%\newcommand{\TextEntry}{\vspace*{0in}}


\begin{enumerate}
  \item Given numerical objects named \texttt{x} and \texttt{y}, calculate this quantity:$ \sqrt{x^2 + y}$ 
\begin{knitrout}
\definecolor{shadecolor}{rgb}{0.969, 0.969, 0.969}\color{fgcolor}\begin{kframe}
\begin{alltt}
\hlkwd{sqrt}\hlstd{(x}\hlopt{^}\hlnum{2} \hlopt{+} \hlstd{y)}
\end{alltt}
\end{kframe}
\end{knitrout}

\item
Load the \texttt{mosaic} and \texttt{mosaicData} packages.  (We will be using the 
\texttt{CPS85} data set from \texttt{mosaicData} for our subsequent examples.)
\begin{knitrout}
\definecolor{shadecolor}{rgb}{0.969, 0.969, 0.969}\color{fgcolor}\begin{kframe}
\begin{alltt}
\hlkwd{require}\hlstd{(mosaic)}
\hlkwd{require}\hlstd{(mosaicData)}
\end{alltt}
\end{kframe}
\end{knitrout}
%  \item Load the data set named ``CPS85" (available through the \texttt{mosaicData} %package), and assign it to an object called \texttt{CPS85}.
%<<message=FALSE,results="hide">>=
%# Remember the quotes around the name of the data set
%CPS85 = fetchData("CPS85")
%@

\item Display the first few rows of the \texttt{CPS85} data frame.


\begin{knitrout}
\definecolor{shadecolor}{rgb}{0.969, 0.969, 0.969}\color{fgcolor}\begin{kframe}
\begin{alltt}
\hlkwd{head}\hlstd{(CPS85)}
\end{alltt}
\end{kframe}
\end{knitrout}

\item Display the names of the variables from the data frame.

\begin{knitrout}
\definecolor{shadecolor}{rgb}{0.969, 0.969, 0.969}\color{fgcolor}\begin{kframe}
\begin{alltt}
\hlkwd{names}\hlstd{(CPS85)}
\end{alltt}
\end{kframe}
\end{knitrout}


\item Calculate (not count by hand!) the number of cases in the data frame.

\begin{knitrout}
\definecolor{shadecolor}{rgb}{0.969, 0.969, 0.969}\color{fgcolor}\begin{kframe}
\begin{alltt}
\hlkwd{nrow}\hlstd{(CPS85)}
\end{alltt}
\end{kframe}
\end{knitrout}


\item Calculate the mean wage of all the people.

\begin{knitrout}
\definecolor{shadecolor}{rgb}{0.969, 0.969, 0.969}\color{fgcolor}\begin{kframe}
\begin{alltt}
\hlkwd{mean}\hlstd{(}\hlopt{~} \hlstd{wage,} \hlkwc{data}\hlstd{=CPS85)}
\end{alltt}
\end{kframe}
\end{knitrout}

\item Calculate the standard deviation of wage for all cases.

\begin{knitrout}
\definecolor{shadecolor}{rgb}{0.969, 0.969, 0.969}\color{fgcolor}\begin{kframe}
\begin{alltt}
\hlkwd{sd}\hlstd{(}\hlopt{~} \hlstd{wage,} \hlkwc{data}\hlstd{=CPS85)}
\end{alltt}
\end{kframe}
\end{knitrout}

\item Calculate the mean wage separately for married and unmarried people.

\begin{knitrout}
\definecolor{shadecolor}{rgb}{0.969, 0.969, 0.969}\color{fgcolor}\begin{kframe}
\begin{alltt}
\hlkwd{mean}\hlstd{(wage} \hlopt{~} \hlstd{married,} \hlkwc{data}\hlstd{=CPS85)}
\end{alltt}
\end{kframe}
\end{knitrout}

\item Create a new variable, \texttt{fraction}, in the data frame that holds the ratio of the person's ``experience" to their age. 


\begin{knitrout}
\definecolor{shadecolor}{rgb}{0.969, 0.969, 0.969}\color{fgcolor}\begin{kframe}
\begin{alltt}
\hlstd{CPS85} \hlkwb{<-} \hlkwd{mutate}\hlstd{(CPS85,} \hlkwc{fraction}\hlstd{=exper}\hlopt{/}\hlstd{age)}
\hlstd{CPS85} \hlkwb{<-} \hlstd{CPS85} \hlopt \hlkwd{mutate}\hlstd{(}\hlkwc{fraction} \hlstd{= exper}\hlopt{/}\hlstd{age)}
\end{alltt}
\end{kframe}
\end{knitrout}

\item Make a box-and-whisker plot of all the people's CPS85.

\begin{knitrout}
\definecolor{shadecolor}{rgb}{0.969, 0.969, 0.969}\color{fgcolor}\begin{kframe}
\begin{alltt}
\hlkwd{bwplot}\hlstd{(}\hlopt{~}\hlstd{wage,} \hlkwc{data}\hlstd{=CPS85)}
\end{alltt}
\end{kframe}
\end{knitrout}

\newpage
\item Make a box-and-whisker plot of the people's wage, but broken down 
by marital status.


\begin{knitrout}
\definecolor{shadecolor}{rgb}{0.969, 0.969, 0.969}\color{fgcolor}\begin{kframe}
\begin{alltt}
\hlkwd{bwplot}\hlstd{(wage} \hlopt{~} \hlstd{married,} \hlkwc{data}\hlstd{=CPS85)}
\end{alltt}
\end{kframe}
\includegraphics[width=\maxwidth]{figure/unnamed-chunk-12} 

\end{knitrout}

\item Make this plot:

\begin{knitrout}
\definecolor{shadecolor}{rgb}{0.969, 0.969, 0.969}\color{fgcolor}
\includegraphics[width=\maxwidth]{figure/unnamed-chunk-13} 

\end{knitrout}

\begin{knitrout}
\definecolor{shadecolor}{rgb}{0.969, 0.969, 0.969}\color{fgcolor}\begin{kframe}
\begin{alltt}
\hlkwd{densityplot}\hlstd{(}\hlopt{~} \hlstd{age,} \hlkwc{groups}\hlstd{=married,} \hlkwc{auto.key}\hlstd{=}\hlnum{TRUE}\hlstd{,} \hlkwc{data}\hlstd{=CPS85)}
\end{alltt}
\end{kframe}
\end{knitrout}
What is different when the command {\tt densityplot( $\sim$ age | married, data=CPS85)} is run?

%<<>>=
%densityplot(~ age | married, data=CPS85) 
%@
\item Calculate (not count by hand!) the number of people by marital status.

\begin{knitrout}
\definecolor{shadecolor}{rgb}{0.969, 0.969, 0.969}\color{fgcolor}\begin{kframe}
\begin{alltt}
\hlkwd{tally}\hlstd{(}\hlopt{~} \hlstd{married,} \hlkwc{data}\hlstd{=CPS85)}
\end{alltt}
\end{kframe}
\end{knitrout}

\item Calculate (not count by hand!) the number of people by marital status and sex simultaneously.


\begin{knitrout}
\definecolor{shadecolor}{rgb}{0.969, 0.969, 0.969}\color{fgcolor}\begin{kframe}
\begin{alltt}
\hlkwd{tally}\hlstd{(}\hlopt{~} \hlstd{married} \hlopt{+} \hlstd{sex,} \hlkwc{data}\hlstd{=CPS85)}
\end{alltt}
\end{kframe}
\end{knitrout}

\end{enumerate}

\end{document}
